\section{Related Work} \label{sec:RelatedWork}

The rapid growth of internet technologies allowed the onset of lots of technologies applied in traceability systems. However, these systems tend to be centralized, monopolistic, and asymmetric. As a consequence, these systems result in trust problems, such as fraud, corruption, tampering, and falsifying information. Likewise, by being a single point of failure, a centralized system is vulnerable to collapse.

Blockchain presents a whole new approach based on decentralization. Nonetheless, by being in its early stages, it has some challenges to deal with, in which scalability and performance become mainly defiance to face the huge amount of data in the real world. Using this technology some solutions have been raised, as follows.

In order to solve some problems with Supply Chain traceability, many internet of things (IoT) technologies, such as RFID and wireless sensor network-based architectures has been applied. However these technologies doesn't guarantee that the information shared by supply chain members in the traceability systems can be trusted \cite{tian2017supply}.

In \cite{tian2017supply}, it is proposed a system that combines haccp (a food safety protocol), Blockchain and IoT in order to provide food safety traceability, involving suppliers, producers, manufacturers, distributors, retailers, consumers and certifiers. Each of these members can add, update and check the information about the product on the Blockchain as long as they register as a user in the system. Each product has also a unique digital cryptographic identifier that connects the physical items to their virtual identity in the system. This virtual identity can be seen as the product information profile.

There are advantages of applying the Blockchain concept to a supply chain. One of the most important is: all stakeholders involved in the supply chain are motivated by the need to demonstrate to customers the superior quality of their methods and products \cite{lu2017adaptable}. 

In addition to serving the functions of a traceability system, a Blockchain can be used as a marketing tool. As Blockchains are fully transparent and participants can control the assets in them, they can be used to enhance image and reputation of a company \cite{van2007essentials}, drive loyalty among existing customers \cite{pizzuti2015global} and attract new ones \cite{svensson2009transparency}. In fact, companies can easily distinguish themselves from competitors by emphasizing transparency and monitoring product flow along the chain. 

The Everledger Diamonds project provides a Blockchain based solution to facilitate tracking from mine to consumer, enabling easier compliance against increasingly strict measures from diamonds produced \cite{crosby2016blockchain}.

IBM Food Trust is a pilot project motivated by food contamination scandals worldwide. The main goal is to tackling food safety in the supply chain using Blockchain technology. This platform tracked pork in China and mangoes in the Americas \cite{kamath2018food}.

These projects are focused on specific products only and are closed projects. Still, there is a general lack of standards for implementation of a Blockchain approach for traceability. A Blockchain must be universal and adaptable to specific situations \cite{valenta2017comparison}. In addition, the need to agree on a particular type of Blockchain to be used puts the parties under pressure. 

Our work is intended to provide a Blockchain based framework in order to facilitate the development of applications for traceability in supply chain management.