\section{Supply Chain Management} \label{sec:General}

Billions of products have being manufactured every day through complex supply chains that can extend to all parts of the world. However, tracing good flows from harvesting and manufacturing to the final consumer is hard. To achieve this flow, we need to improve Supply Chain visibility \cite{galvez2018future}. %Before reaching the end consumer, the goods go through an often wide network of retailers, distributors, carriers, warehousing facilities and suppliers who participate in the design, production, delivery and sales process of a product, but in many cases. These steps are a dimension invisible to the consumer \cite{provenance2015}.
%Traceability is the ability to preserve the product identities, their origins, and transformations, so that the collection, documentation and maintenance of information related to all processes in the production chain must be ensured \cite{gryna1998juran, Opara2001} .

Traceability is one of the key challenges encountered in the business world, with most companies having little or no information about their own second and third-tier suppliers. Transparency and end-to-end visibility of the supply chain can help shape product, raw material, test control, and end product flow, enabling better operations and risk analysis to ensure better chain productivity \cite{abeyratne2016blockchain}.

%inicio do general context antigo
Traceability systems typically store information in standard databases controlled by service providers. This centralized data storage becomes a single point of failure and tampering risks. As a consequence, these systems result in trusting problems, such as fraud, corruption and tampering. Likewise, as a single point of failure, a centralized system is vulnerable to collapse \cite{tian2017supply}.

Nowadays, Blockchain presents a whole new approach based on decentralization, enabling end-to-end traceability, allowing consumers to access the asset's history of these products through a software application \cite{galvez2018future}.

SCM requires to control who can write and read data to/from the Blockchain. In order to do that, the first step is identity. In the SCM context, the peers are known and the system needs to know who a user is, to define rules about what data they can commit, and what data they can consume from the ledger. So, in a corporate case scenario, Blockchain for the business, Blockchain for supply value chains, a private Blockchain provides this needed characteristic.